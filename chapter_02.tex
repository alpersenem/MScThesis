%%%%%%%%%%%%%%%%%%%%%%%%%%%%%%%%%%%%%%%%%%%%%%%%%%%%%%%%%%%%%%%%%
\chapter{CONDITION MONITORING OF INDUCTION MOTORS: BACKGROUND }\label{Ch2}
%%%%%%%%%%%%%%%%%%%%%%%%%%%%%%%%%%%%%%%%%%%%%%%%%%%%%%%%%%%%%%%%%
\vspace*{-12pt} % If no text above section, use this vspace* to lift the whole part to the proper starting point - SBÖ
\section{Introduction of Induction Motors}
\subsection{Principle of operation}

Electric motors are divided into two classes depending on their power supply type: direct current (DC) or alternating current (AC). The latter can be broken into two classes as synchronous or induction according to their operating speed. Induction motors, which operates slightly lower than synchronous speed, are also sub-divided as wounded and squirrel-cage motors. In this study, squirrel-cage induction motors have been investigated by means of induction motors, since the squirrel-cage type is predominantly used in industrial applications. 

Induction motors run at a speed slightly lower than synchronous speed at the point where motor torque and load torque are equal \cite{gunnar2016}. The difference between the actual speed and synchronous speed is known as slip \cite{doe2008improving}.

synchronous speed equation comes here.

slip equation comes here.

In Principle, induction motors transfer electrical energy into mechanical energy by interlinking two electrical components: stator as stationary part and rotor as rotational part. Electrical energy transmitted from stator to rotor via electromagnetic induction, then a mechanical component bearing guides rotor to provide mechanical power \cite{oliver1992electric,karmakar2016induction}.

motor diagram comes here.

\subsection{VFD-fed induction motors}

two figures will be added from iet 108.

A variable frequency drive, also called as adjustable-frequency drive (AFD), variable-speed drive (VSD) or inverter, fed motor system controls the rotation speed of the induction motor by controlling the supply frequency and voltage of the motor. The main difference between line-start and VFD-fed induction motors is that while in line-start mode supply voltage is the only controllable parameter, on the other hand, VFD-fed has the ability to control torque and speed easily \cite{faiz2017fault}.

From a historical point of view, DC motors have been utilised in speed control applications. However, as a result of advances in power semiconductor technology used in inverters, the performance of AC motors in terms of precision, response, and speed range began to exceed that of DC motors \cite{doe2008improving,mikami2011historical}. As a driving force behind the induction motor control dominance today, VFDs generally have the following control strategies regarding speed and torque regulation \cite{weg,danfoss}:

\begin{itemize}
	\item Voltage per Frequency Control (V/f)
	\item Field Oriented Control (FOC)
	\item Direct Torque Control (DTC)
\end{itemize}

The common idea behind these methods is based on controlling the torque and flux references applied to the motor separately, as in DC motor control \cite{faiz2017fault}. In the scope of this thesis, only the V/f control strategy emphasized due to the widespread adoption of the control method in pump, compressor and fan applications. 

V/f control can be employed in both open-loop and closed-loop modes. Open-loop V/f control, which is by far the most popular control due to its simplicity, as the name implies, creates a constant air-gap flux by keeping the ratio between the voltage and frequency applied to the induction motor constant, and as a result, it provides the opportunity to work at operating frequencies from zero to nominal frequency \cite{bose2002modern}. 

VFDs come with benefits such that energy savings, reliability and product quality, yet in concern of fault diagnosis they introduce a number of factors, which will be discussed later on, that increase the complexity. 

\subsection{Need for condition monitoring}

Condition monitoring defined as measuring activities concerning characteristics and parameters of physical equipment at predetermined intervals either manually or automatically \cite{en201713306}. Leveraging rapid technological advancements in data storage, data process and network structure, condition monitoring became one of the driving force behind the industry 4.0 paradigm. The key goal behind this paradigm is to acquisition, transmission and analysis of data in order to predict future behaviours of machinery, or plant on a larger scale, to boost efficiency and reliability \cite{lughofer2019prologue,RUIZSARMIENTO}. 

Researchers from both academia and industry have devoted significant attention to condition monitoring of induction motors over decades. Even though induction motors renowned for robustness, environmental, electrical and mechanical effects may lead induction motors to failure. As a result, industrial processes subjected to potential losses in a manner of time and capital, so the desire to minimize or even prevent these losses emerges the need for condition monitoring. 

\subsection{Maintenance strategies}

Maintenance can be defined as the combination of all technical and administrative actions taken to maintain or restore an item throughout its life cycle in a condition where it can fulfil its designed function \cite{en201713306}. A motor maintenance program should effectively address reliability, cost, and scheduling issues, as well as the causes of the most common motor failures. Essentially, there are two types of maintenance strategies: corrective and preventive. 

bsi map comes here as figure.

Corrective maintenance is a type of maintenance performed after the induction motor failure to detect the fault and restore it to operational condition \cite{en201713306}. The main purpose of this type of maintenance is to get the equipment up and running as soon as possible by repairing or replacing the defective equipment. However, corrective maintenance as a failure-driven method contains a high-risk potential as faults may occur at unexpected times, can disrupt the operation. Since this type of maintenance approach does not take into account the damages that may occur, it may be suitable for equipment that is not critical to the business that does not pose a safety risk.

Preventive maintenance, on the other hand, aims to detect faults at an early stage and correct them before they introduce risk to operation \cite{en201713306}. Preventive maintenance employed to increase efficiency and reliability by taking into account the probability of failure or the ageing of the equipment, at certain intervals or according to pre-planned scheduling. Although this approach is beneficial in cases where the wear-out characteristics are evident, it has disadvantages, especially in terms of not being able to use equipment lifespan efficiently and increasing the maintenance cost compared to the corrective maintenance approach \cite{AHMAD}.

Predictive maintenance is a condition-based approach to maintenance that is used to evaluate the parameters and characteristics of the equipment or to make predictions based on repeated analysis \cite{en201713306}. Compared to preventive maintenance, predictive maintenance maximizes equipment service-life whilst minimizing unnecessary maintenance. In 99\% of machine failures, it is possible to observe indications that malfunctions will occur, in other words, the necessary measures can be taken before 99\% of the faults occur by continuously monitoring the machine \cite{AHMAD}.

Under the predictive maintenance approach, decision-making can be divided into two: diagnosis, which is the analysis of the current situation, and prognosis, which is the assessment of conditions measured over time \cite{tinga2019physical}. A P-F curve can be used to better understand diagnostic and prognostic monitoring systems.

PF diagram comes here.

The downside of predictive maintenance is that it requires additional equipment and relatively high investment costs. But the advantage of VFDs also comes out here. As they currently monitor motor parameters in control applications, they have a high potential for predictive maintenance applications without the need for additional sensors and investments.

% \vspace{6pt}
% \begin{figure}[h]
% 	\centering
% 	\includegraphics[scale=.3]{./fig/sekil1}
% 	% sekil1.eps: 0x0 pixel, 300dpi, 0.00x0.00 cm, bb=14 14 592 479
% 	\vspace{6pt}
% 	\caption{All tables and figures must be horizontally centered on the page.}
% 	\label{Figure2.1}
% \end{figure}

% %\begin{figure}
% %	\begin{minipage}[b]{.5\linewidth}
% %		\centering
% %		\includegraphics[scale=.2]{./fig/sekil1}
% %		\subcaption{A subfigure}\label{Figure2.2a}
% %	\end{minipage}
% %	\begin{minipage}[b]{.5\linewidth}
% %		\centering
% %		\includegraphics[scale=.2]{./fig/sekil1}
% %		\subcaption{Another subfigure}\label{Figure2.2b}
% %	\end{minipage}
% %	\caption{A figure}\label{Figure2.2} % If no need a caption for main figure comment it out 
% %\end{figure}
% %%Figure letter: \subref{Figure2.2a}

% %\begin{figure}
% %	\begin{subfigure}[b]{.5\linewidth}
% %		\centering
% %		\includegraphics[scale=.2]{./fig/sekil1}
% %		\caption{A subfigure}\label{Figure2.3a}
% %	\end{subfigure}
% %	\begin{subfigure}[b]{.5\linewidth}
% %		\centering
% %		\includegraphics[scale=.2]{./fig/sekil1}
% %		\caption{Another subfigure}\label{Figure2.3b}
% %	\end{subfigure}
% %	\caption{A figure}\label{Figure2.3}
% %\end{figure}

% % Subfigure example with proper LOF usage - SBÖ
% \begin{figure}[h]
% 	\centering
% 	\begin{subfigure}{.8\textwidth}
% 		\centering
% 		\includegraphics[scale=.3]{./fig/sekil1}
% 		\firstsubcaption{First subcaption of the subfigure.}
% 		\label{Figure2.2a}
% 	\end{subfigure}
% 	\begin{subfigure}{.8\textwidth}
% 		\centering
% 		\includegraphics[scale=.3]{./fig/sekil1}
% 		\nextsubcaption{Second subcaption of the subfigure.}
% 		\label{Figure2.2b}
% 	\end{subfigure}
%     \caption{An example of subfigure main caption.}\label{Figure2.2}
% \end{figure}

% %\begin{figure}
% %	\centering     % not \center
% %	\subcaption[]{Another subfigure}{\label{fig:a}\includegraphics[scale=.2]{./fig/sekil1}}
% %	\subcaption[]{Another subfigure}{\label{fig:b}\includegraphics[scale=.2]{./fig/sekil1}}
% %	%\caption{(a) this is fig 1 (b) this is fig 2.}
% %	\label{L50}
% %\end{figure}

% In Figure \ref{Figure2.2}, sed diam nonumy eirmod tempor invidunt ut labore et dolore magna aliquyam erat, sed diam voluptua. At vero eos et accusam et justo duo dolores et ea rebum. Lorem ipsum dolor sit amet, consetetur sadipscing elitr, sed diam nonumy eirmod tempor invidunt ut labore et dolore magna aliquyam erat, sed diam voluptua. At vero eos et accusam et justo duo dolores et ea rebum. At vero eos et accusam et justo duo dolores et ea rebum. At vero eos et accusam et justo duo dolores et ea rebum. At vero eos et accusam et justo duo dolores et ea rebum. At vero eos et accusam et justo duo dolores et ea rebum. At vero eos et accusam et justo duo dolores et ea rebum. At vero eos et accusam et justo duo dolores et ea rebum in Figure \ref{Figure2.2a}.

% \begin{figure}
% 	\centering
% 	\includegraphics[width=10cm,keepaspectratio=true]{./fig/sekil2}
% 	% sekil2.eps: 0x0 pixel, 300dpi, 0.00x0.00 cm, bb=14 14 818 556
% 	\vspace{3pt}
% 	\caption{Example figure.}
% 	\label{Figure2.3}
% \end{figure}
% \vspace{-6pt}
\section{Induction Motor Fault Types}

From a mechanical perspective, induction machines basically consist of three components: stator, rotor and bearing. Electrical, mechanical, and environmental disturbances constantly affect asynchronous motor components and cause most malfunctions \cite{bonnet2010}. Table \ref{Table2.1} exhibits various surveys that studied and categorized the most common failures \cite{motor1985report,albrecht1986assessment,albrecht1987assessment,thorsen1995survey,bonnett2008increased}. 

\begin{table*}[h]
	{\setlength{\tabcolsep}{12pt}
		\caption{Table with single row and centered columns.}
		\begin{center}
			\vspace{-6mm}
			\begin{tabular}{ccccc}
				\hline \\[-2.45ex] \hline \\[-2.1ex]
				Component & IEEE & EPRI & Thorsen-Dalva & Bonnet-Yung \\
				\hline \\[-1.8ex]
				Bering & 44 & 41 & 51 & 69 \\
				Stator & 26 & 37  & 16 & 21\\
				Rotor & 8 & 10 & 5 & 7 \\
				Other & 22 & 12  & 28 & 3\\
				\hline
			\end{tabular}
			\vspace{-6mm}
		\end{center}
		\label{Table2.1}}
\end{table*}

As can be seen in Table \ref{Table2.1}, most of the faults associated with bearings followed by stator related faults. It also should be noted that these surveys do not include the effects of power electronics. A motor controlled by VFD is subjected to short and high voltage pulses called PWM (Pulse Width Modulation), which are sent at a very high frequency, which can have a detrimental effect on the wire insulation and cause a burn on the stator \cite{gunnar2016}. Although this problem can be solved with high-quality insulation, PWM signals also create non-continuous electrical discharges on the bearings, causing wear which reduces bearing lifespan \cite{trigeassou2013electrical}. Therefore, it would not be wrong to conclude that bearing and stator failures will also have a high rate in VFD-fed induction motors.

\subsection{Bearing related faults}

In all kinds of electrical machines, the mechanical element positioned between the frame that initiates the movement and the rotating axis shaft is called a bearing. These mechanical elements, which help the rotational movement of the electric motor, are exposed to many internal and external destructive effects during their operation and failures arise as a result. Major sources of bearing failures are given below \cite{zhang2010survey,easa,skf,schoen1995motor,en201320958,bonnet2010}:

\textbf{Mechanical stresses}: Fatigue, which mostly begins on the surface, turns into small-sized material ruptures at the beginning and later dimensional surface indentations and protrusions. Loose motor connection, misalignment where the motor shaft and load shaft are connected without aligning on the same axis, angular misalignment where the motor shaft and load shaft axes are connected at a certain angle, and unbalanced load connection, which is an unbalance condition where the centre of gravity of the load connected to the motor shaft is not on the rotation axis are other mechanical disturbances on the bearing.

\textbf{Environmental stresses}: Corrosion occurs on the bearing surfaces used in high humidity working environments. Especially the moisture absorbed in the bearing oil initiates this process and the rust that occurs due to corrosion causes deterioration that turns into indentation and protrusion on the surface of the bearing element, and cracks in the later stages.

\textbf{Thermal stresses}: Insufficient lubrication generally causes problems with bearing components. Normally, there is a layer of oil in the bearing that prevents direct contact between the rotating elements so that their surfaces do not wear out quickly. In case of insufficient lubrication, excessive wear and subsequent material deterioration occur as a result of increased friction due to direct contact between metal surfaces.

\textbf{Electrical stresses}: the electrical discharge current effect occurs with a fault current flowing through the bearings from the motor frame to the ground in motors that do not have a suitable ground connection. Asymmetry of stator windings, permanent magnetism effect developing in the motor over time, electrostatic charge accumulation in the motor frame and application of voltage to the motor shaft from the outside, or common end voltages generated due to the high switching frequency of semiconductor power electronics (VFDs using PWM) are the factors that cause this malfunction. The irregular current will cause wear and tear on the bearing metal surface, and as a result, the degree of material rupture and surface deterioration increases.

Vibration in the motor causes the rotor to rotate irregularly or axially unbalanced in the motor air gap. Any axial misalignment that occurs in the motor air gap adversely affects the air gap flux density and causes the formation of harmonic components \cite{schoen1995motor,en201320958,faiz2017fault}. Consequently, this can induce harmonic components in the current drawn by the motor with frequencies given by formula \cite{schoen1995motor}:
\begin{equation}
	f_{bng}=f_{e} \pm m f_{\mathrm{v}}
	\label{bearingfault}
\end{equation}
where,\\
$f_{e} \quad$ is the electrical supply frequency;\\
$f_{\mathrm{v}} \quad$ is the rotational speed frequency of the rotor;\\
$\mathrm{m} \quad$ is the harmonic number $1,2,3 \ldots$..;\\
$f_{bng} \quad$ is the current component frequency due to air gap changes.

\subsection{Stator related faults}

As researches have shown, stator faults occupy an important place among asynchronous motor faults after bearing \cite{motor1985report,albrecht1986assessment,albrecht1987assessment,thorsen1995survey,bonnett2008increased}. Mechanical, electrical, thermal and environmental factors cause malfunctions in the stator windings, as well as their laminations \cite{karmakar2016induction,Siddique}. Winding faults, as the most common stator faults, are winding short-circuit faults that are mostly the result of the aforementioned effects of the winding insulation. Types of winding faults are as follows \cite{karmakar2016induction,Siddique,lipo}:

\begin{itemize}
	\item Short-circuit between two turns in the same phase, (turn-turn failure)
	\item Short-circuit between two coils side by side in the same phase (coil-coil failure),
	\item Short-circuit between the turns of two phases (phase-phase failure),
	\item Short circuit consisting of all three-phase turns,
	\item Short-circuit between the conductor of the winding and the stator core (phase-ground short circuit),
	\item Open-circuit fault when winding gets break.
\end{itemize}

The factors that cause the motor winding insulation to deteriorate are explained below \cite{karmakar2016induction,Siddique,faiz2017fault}:

\textbf{Mechanical stresses}: While the motor is running, the rotor may rub or hit the inner surface of the stator due to motor shaft deterioration, bearing failures and misalignment. This force creates a turn-to-turn or a phase-to-earth short-circuit, causing the stator coil and the stator winding insulation to break down. On the other hand, winding breakage may occur due to vibration during operation and therefore the motor produces the open-circuit fault.

\textbf{Environmental stresses}: The environment in which the motor is running can be very hot, cold or humid. On the other hand, substances in the external environment can contaminate the windings, causing the heat dissipation to deteriorate and the insulation to be damaged. In addition, the airflow can be blocked and cannot absorb the air required for cooling. Therefore, it causes the motor windings to heat and consequently the insulation to deteriorate.

\textbf{Thermal stresses}: Thermal effects occur as a result of overloading or a motor failure. With motor overload, the motor temperature rises above the limit value of the insulation class and the insulation deteriorates. At this point, every 3.5\% unbalance in the motor supply voltage increases the temperature of the motor by 10°C. In addition, every 10°C temperature increase above the limit temperature value of the insulation halves the life of the insulation.

\textbf{Electrical stresses}: The main reason for this is sudden changes in supply voltage. Transients during commissioning and decommissioning and voltage fluctuations frequently occur, especially in asynchronous motors powered by variable frequency drives. Winding insulations deteriorate due to these voltage variations.

Under the inter-turn short-circuit condition, a significant deviation in rotor slot harmonics components,called as principle slot harmonics (PSH), occurs and can be obtained by given formula \cite{Penman};

\begin{equation}
	f_{st}=f_{e} \cdot\left[n \cdot \frac{(1-s)}{p} \pm k\right]
	\label{statorfault}
\end{equation}
where,\\
$f_{e} \quad$ is the electrical supply frequency ;\\
$\mathrm{p} \quad$ is the number of pole pairs of the motor ;\\
$\mathrm{n} \quad$ = $1,2,3 \ldots (2p-1)$;\\
$\mathrm{s} \quad$ is the slip ;\\
$\mathrm{k} \quad$ is the harmonic number $1,2,3 \ldots$..;\\
$f_{st} \quad$ is the principle slot harmonic frequencies.

\subsection{Rotor related faults}





% % Change margins on the fly to reset the page margins to one inch - SBÖ
% \newenvironment{changemargin}[4]{
% 	\begin{list}{}{
% 			\setlength{\voffset}{#1}
% 			\setlength{\oddsidemargin}{#2}
% 			\setlength{\evensidemargin}{#3}
% 			\setlength{\textheight}{#3}
% 		}
% 		\item[] ~ \par
% 		% Get rid of the extra space inserted by the previous line
% 		%\vspace*{-2em}
% 	}{
% 	\end{list}
% }

% % All the figures and also odd page figures normally face inside the thesis, however the rule requires figures always face to the right. - SBÖ
% % Figures on landscape pages has to be centered and facing to the right (ITU) - SBÖ
% \begin{landscape}
% 	\thispagestyle{empty} %Remove the bottom page numbering
% %	\begin{changemargin}{-0.4mm}{0mm}{0mm} %Set all the margins to zero - SBÖ
% 	%\thispagestyle{lscape}
% 	\vspace*{5mm}
% 	\begin{figure*}[ht]
% 		\centering
% 		%\begin{tabular}{@{}cc@{}}
% 		\includegraphics[scale=.41,keepaspectratio=true]{./fig/sekil3} %&
% 		%\includegraphics[width=50mm]{./fig/sekil3}
% 		%\end{tabular}                                       
% 		\caption{Landscape-oriented, full-page figure.}
% 		\label{Figure2.4}
% 	\end{figure*}
	
% % Set the page number on the right side for odd numbered pages
%       \begin{tikzpicture}[remember picture, overlay]
% 		\node[xshift=-25mm+148.5mm, yshift=17mm-210mm+15mm] (number) at (current page text area.east) {\thepage};
% 	  \end{tikzpicture}
	  
% %\end{changemargin}
% \end{landscape}

% % All the figures and also even page figures normally face inside the thesis, however the rule requires figures always face to the right. - SBÖ
% % Figures on landscape pages has to be centered and facing to the right (ITU) - SBÖ
% \begin{landscape}
% 	\thispagestyle{empty} % Remove the bottom page numbering
% %	\begin{changemargin}{-0.4mm}{0mm}{0mm} %Set all the margins to zero - SBÖ
% 		%\thispagestyle{lscape}

% 		\vspace*{20mm}
% 		\begin{figure*}[ht]
% 			\centering
% 			%\begin{tabular}{@{}cc@{}}
% 				\includegraphics[scale=.41,keepaspectratio=true]{./fig/sekil3} %&
% 				%\includegraphics[width=50mm]{./fig/sekil3}
% 			%\end{tabular}                                       
% 			\caption{Landscape-oriented, full-page figure.}
% 			\label{Figure2.5}
% 		\end{figure*}
	   
% % Set the page number on the left side for even numbered pages
% 		%\begin{tikzpicture}[remember picture, overlay]
% 		% \node[xshift=-25mm+148.5mm, yshift=-1mm-15mm, rotate=180] (number) at (current page text area.east) {\thepage};
% 		%\end{tikzpicture}
		
% % Set the page number on the right side for even numbered pages as well
% 		\begin{tikzpicture}[remember picture, overlay]
% 		 \node[xshift=-25mm+148.5mm, yshift=17mm-210mm] (number) at (current page text area.east) {\thepage};
% 		\end{tikzpicture}
		
% %	\end{changemargin}
% \end{landscape}

%\newpage
\section{Condition Monitoring Techniques}

\subsection{Temperature monitoring}
\subsection{Vibration monitoring}
\subsection{Motor current monitoring}


Lorem ipsum dolor sit amet, consetetur sadipscing elitr, sed diam nonumy eirmod tempor invidunt ut labore et dolore magna aliquyam erat, sed diam voluptua. At vero eos et accusam et justo duo dolores et ea rebum. Stet clita kasd gub rgren, no sea takimata sanctus est Lorem ipsum dolor sit amet, consetetur sadipscing elitr, sed diam nonumy eirmod tempor invidunt ut lab ore sit et dolore magna.

% \begin{table*}[h]
% 	{\setlength{\tabcolsep}{14pt}
% 		\caption{Table with single row and centered columns.}
% 		\begin{center}
% 			\vspace{-6mm}
% 			\begin{tabular}{cccc}
% 				\hline \\[-2.45ex] \hline \\[-2.1ex]
% 				Column A & Column B & Column C & Column D \\
% 				\hline \\[-1.8ex]
% 				Row A & Row A & Row A & Row A \\
% 				Row B & Row B & Row B & Row B \\
% 				Row C & Row C & Row C & Row C \\
% 				\hline
% 			\end{tabular}
% 			\vspace{-6mm}
% 		\end{center}
% 		\label{Table2.1}}
% \end{table*}

As seen in Table \ref{Table2.1}, lorem ipsum dolor sit amet, consetetur sadipscing elitr, sed diam nonumy eirmod tempor invidunt ut labore et dolore magna aliquyam erat, sed diam voluptua. At vero eos et accusam et justo duo dolores et ea rebum. Stet clita kasd gub rgren, no sea takimata sanctus est Lorem ipsum dolor sit amet, consetetur sadipscing elitr, sed diam nonumy eirmod tempor invidunt ut lab ore sit et dolore magna.

% \begin{table*}[h]
% 	{\setlength{\tabcolsep}{14pt}
% 		\caption{Table captions must be ended with a full stop.}
% 		\begin{center}
% 			\vspace{-6mm}
% 			\begin{tabular}{cccc}
% 				\hline \\[-2.45ex] \hline \\[-2.1ex]
% 				Column A & Column B & Column C & Column D \\
% 				\hline \\[-1.8ex]
% 				Row A & Row A & Row A & Row A \\
% 				Row B & Row B & Row B & Row B \\
% 				Row C & Row C & Row C & Row C \\
% 				\hline
% 			\end{tabular}
% 			\vspace{-6mm}
% 		\end{center}
% 		\label{Table2.2}}
% \end{table*}

Lorem ipsum dolor sit amet, consetetur sadipscing elitr, sed diam nonumy eirmod tempor invidunt ut labore et dolore magna aliquyam erat, sed diam voluptua. At vero eos et accusam et justo duo dolores et ea rebum, as seen in Table \ref{Table2.2}. 

Lorem ipsum dolor sit amet, consetetur sadipscing elitr, sed diam nonumy eirmod tempor invidunt ut labore et dolore magna aliquyam erat, sed diam voluptua. At vero eos et accusam et justo duo dolores et ea rebum. Stet clita kasd gub rgren, no sea takimata sanctus est Lorem ipsum dolor sit amet, consetetur sadipscing elitr, sed diam nonumy eirmod tempor invidunt ut lab ore sit et dolore magna. Lorem ipsum dolor sit amet, consetetur sadipscing elitr, sed diam nonumy eirmod tempor invidunt ut labore et dolore magna aliquyam erat, sed diam voluptua. At vero eos et accusam et justo duo dolores et ea rebum \cite{Roberts_Jackson_1991}. 

\section{Signal Processing Techniques}
\subsection{Time domain based signal analysis}
\subsubsection{Higher order statistics}
\subsection{Time-frequency based signal analysis}
\subsubsection{Wavelet Transform}
\subsection{Frequency based signal analysis}
\subsubsection{Shannon-Nyquist sampling theory}
\subsubsection{Fast Fourier transform}
\subsubsection{Power spectral density estimation}

Lorem ipsum dolor sit amet, consetetur sadipscing elitr, sed diam nonumy eirmod tempor invidunt ut labore et dolore magna aliquyam erat, sed diam voluptua. At vero eos et accusam et justo duo dolores et ea rebum. Stet clita kasd gub rgren, no sea takimata sanctus est Lorem ipsum dolor sit amet, consetetur sadipscing elitr, sed diam nonumy eirmod tempor invidunt ut lab ore sit et dolore magna.

% ---------------------------------------------------------------- %
% Page numbers must be on the bottom-middle of short side (when    %
% portrait-oriented), or bottom-middle of long side (when	       %
% landscape-oriented)						                       %
% ---------------------------------------------------------------- %
% Odd page landscape table and page numbering - SBÖ		
\begin{landscape}
	\thispagestyle{empty}
%	\vspace*{-6mm}
%	\begin{changemargin}{0.4mm}{0mm}{0mm} %Set all the margins to zero - SBÖ
	\begin{table*}[htb!]
		{\setlength{\tabcolsep}{14pt}
			%\hspace*{5mm}
			%\vspace*{-6mm}
			\caption{Prof. Dr. Galip TEPEHAN \,\, Captioning in landscape-oriented pages:
				the most important aspect is to align the lines horizontally.}
			\begin{center}
				\vspace{-6mm}
				\begin{tabular}{lccrrrrr}
					\hline\hline
					\multirow{2}{*}{Parametre} & \multirow{2}{*}{Column 2} & \multirow{2}{*}{Column 3} & \multicolumn{3}{c|}{Column 4} & \multicolumn{2}{c}{Column 5}\\ \cline{4-8}
					& & & Subcolumn & Subcolumn & Subcolumn & Subcolumn & Subcolumn\\
					\hline
					Row 1 & -7.680442 & 7.6986348 & 0.00 & 0.00 & 0.00 & 12 & 12 \\
					Row 2 & 140 & - & 0.50 & 0.00 & 0.00 & 0 & 0 \\
					Row 3 & 37.174357 & 37.16192697 & 0.00 & 0.00 & 0.00 & 0 & 24 \\
					Row 4 & 140 & - & 0.50 & 0.00 & 0.00 & 0 & 0 \\
					Row 5 & 37.174357 & 37.16192697 & 0.00 & 0.00 & 0.00 & 0 & 24 \\
					Row 6 & 140 & - & 0.50 & 0.00 & 0.00 & 0 & 0 \\
					Row 7 & 37.174357 & 37.16192697 & 0.00 & 0.00 & 0.00 & 0 & 24 \\
					Row 8 & 140 & - & 0.50 & 0.00 & 0.00 & 0 & 0 \\
					Row 9 & 37.174357 & 37.16192697 & 0.00 & 0.00 & 0.00 & 0 & 24 \\
					Row 10 & 140 & - & 0.50 & 0.00 & 0.00 & 0 & 0 \\
					Row 11 & 37.174357 & 37.16192697 & 0.00 & 0.00 & 0.00 & 0 & 24 \\
					Row 12 & 140 & - & 0.50 & 0.00 & 0.00 & 0 & 0 \\
					Row 13 & 37.174357 & 37.16192697 & 0.00 & 0.00 & 0.00 & 0 & 24 \\
					Row 14 & 140 & - & 0.50 & 0.00 & 0.00 & 0 & 0 \\
					Row 15 & 37.174357 & 37.16192697 & 0.00 & 0.00 & 0.00 & 0 & 24 \\
					\hline
				\end{tabular}
			\end{center}
			\begin{center}
				\label{Table2.3}
			\end{center}
		}
	\end{table*}
% Set the page number on the right side for odd numbered pages
		\begin{tikzpicture}[remember picture,overlay]
		\node[xshift=-10mm+148.5mm, yshift=2mm-210mm+30mm] (number) at (current page text area.east) {\thepage};
		\end{tikzpicture}
%   \end{changemargin}
\end{landscape}

% ---------------------------------------------------------------- %
% Page numbers must be on the bottom-middle of short side (when    %
% portrait-oriented), or bottom-middle of long side (when	       %
% landscape-oriented)						                       %
% ---------------------------------------------------------------- %
% Even page landscape table and page numbering - SBÖ		
\begin{landscape}
	\thispagestyle{empty}
	%\vspace*{-6mm}
%	\begin{changemargin}{0.4mm}{0mm}{0mm} %Set all the margins to zero - SBÖ
		\begin{table*}[htb!]
			{\setlength{\tabcolsep}{14pt}
				%\hspace*{5mm}
				%\vspace*{-6mm}
				\caption{Prof. Dr. Galip TEPEHAN \,\, Captioning in landscape-oriented pages:
					the most important aspect is to align the lines horizontally.}
				\begin{center}
					\vspace{-6mm}
					\begin{tabular}{lccrrrrr}
						\hline\hline
						\multirow{2}{*}{Parametre} & \multirow{2}{*}{Column 2} & \multirow{2}{*}{Column 3} & \multicolumn{3}{c|}{Column 4} & \multicolumn{2}{c}{Column 5}\\ \cline{4-8}
						& & & Subcolumn & Subcolumn & Subcolumn & Subcolumn & Subcolumn\\
						\hline
						Row 1 & -7.680442 & 7.6986348 & 0.00 & 0.00 & 0.00 & 12 & 12 \\
						Row 2 & 140 & - & 0.50 & 0.00 & 0.00 & 0 & 0 \\
						Row 3 & 37.174357 & 37.16192697 & 0.00 & 0.00 & 0.00 & 0 & 24 \\
						Row 4 & 140 & - & 0.50 & 0.00 & 0.00 & 0 & 0 \\
						Row 5 & 37.174357 & 37.16192697 & 0.00 & 0.00 & 0.00 & 0 & 24 \\
						Row 6 & 140 & - & 0.50 & 0.00 & 0.00 & 0 & 0 \\
						Row 7 & 37.174357 & 37.16192697 & 0.00 & 0.00 & 0.00 & 0 & 24 \\
						Row 8 & 140 & - & 0.50 & 0.00 & 0.00 & 0 & 0 \\
					\end{tabular}
				\end{center}
				\begin{center}
					\label{Table2.4}
				\end{center}
			}
		\end{table*}
% Set the page number on the right side for even numbered pages
		\begin{tikzpicture}[remember picture,overlay]
		\node[xshift=-25mm+148.5mm, yshift=2mm-210mm+15mm] (number) at (current page text area.east) {\thepage};
		\end{tikzpicture}
%	\end{changemargin}
\end{landscape}

\begin{table}[!htbp] \centering
	\caption{ Neighborhoods Visited }
	\vspace{-3mm}
	\label{}
	\begin{tabular}{@{\extracolsep{5pt}} llrrr} 
	\\[-1.8ex]\hline 
		\hline \\[-1.8ex] 
		\multicolumn{1}{c}{Variable} & \multicolumn{1}{c}{Values} & \multicolumn{1}{c}{Count} & \multicolumn{1}{c}{\%} & \multicolumn{1}{c}{Cum. \%} \\
		\hline \\[-1.8ex] 
		\multirow{ 4 }{*}{ visit }  &  FALSE  &  2  &  33.33  &  33.33  \\
		\hhline{}  &  TRUE  &  3  &  50.00  &  83.33  \\
		\hhline{}  &  NA  &  1  &  16.67  &  100.00  \\
	    \hhline{}  &  Total  &  6  &  100.00  &    \\
		\hline \\[-1.8ex] 
	\end{tabular}
\end{table}

% Multi-page longtable example spreading couple of pages - SBÖ
\begin{center}
	\begin{longtable}{ccc}
		%Here is the caption, the stuff in [] is the table of contents entry,
		%the stuff in {} is the title that will appear on the first page of the
		%table.
		\caption[Feasible triples for a highly variable Grid]{Feasible triples
			for highly variable Grid, MLMMH.} \label{Table2.6} \vspace{-1.75mm}\\
		%This is the header for the first page of the table...
		\hline\\[-2.45ex] \hline \\[-1.8ex] % Distancing of the hlines adjausted from the text 
		\multicolumn{1}{c}{{Time (s)}} &
		\multicolumn{1}{c}{{Triple chosen}} &
		\multicolumn{1}{c}{{Other feasible triples}} \\[0.5ex] \hline
		\\[-1.8ex]
		\endfirsthead
		
		%This is the header for the remaining page(s) of the table...
		\multicolumn{3}{c}{{\tablename} \textbf{\thetable{}} \textbf{(continued) :} Feasible triples
			for highly variable Grid, MLMMH.} \\[0.5ex]
		\hline\\[-2.45ex] \hline \\[-1.8ex]
		\multicolumn{1}{c}{{Time (s)}} &
		\multicolumn{1}{c}{{Triple chosen}} &
		\multicolumn{1}{c}{{Other feasible triples}} \\[0.5ex] \hline
		\\[-1.8ex]
		\endhead
		
		%This is the footer for all pages except the last page of the table...
		%\multicolumn{3}{l}{{Continued on Next Page\ldots}} \\
		\\[-1.8ex] \hline
		\endfoot
		
		%This is the footer for the last page of the table...
		\\[-1.8ex] \hline
		\endlastfoot
		
		%Now the data...
		0      & (1, 11, 13725) & (1, 12, 10980), (1, 13, 8235), (2, 2, 0), (3, 1, 0) \\
		2745   & (1, 12, 10980) & (1, 13, 8235), (2, 2, 0), (2, 3, 0), (3, 1, 0) \\
		5490   & (1, 12, 13725) & (2, 2, 2745), (2, 3, 0), (3, 1, 0) \\
		8235   & (1, 12, 16470) & (1, 13, 13725), (2, 2, 2745), (2, 3, 0), (3, 1, 0) \\
		% <data removed>
		164700 & (1, 13, 13725) & (2, 2, 2745), (2, 3, 0), (3, 1, 0) \\
		0      & (1, 11, 13725) & (1, 12, 10980), (1, 13, 8235), (2, 2, 0), (3, 1, 0) \\
		2745   & (1, 12, 10980) & (1, 13, 8235), (2, 2, 0), (2, 3, 0), (3, 1, 0) \\
		5490   & (1, 12, 13725) & (2, 2, 2745), (2, 3, 0), (3, 1, 0) \\
		8235   & (1, 12, 16470) & (1, 13, 13725), (2, 2, 2745), (2, 3, 0), (3, 1, 0) \\
		% <data removed>
		164700 & (1, 13, 13725) & (2, 2, 2745), (2, 3, 0), (3, 1, 0) \\
		0      & (1, 11, 13725) & (1, 12, 10980), (1, 13, 8235), (2, 2, 0), (3, 1, 0) \\
		2745   & (1, 12, 10980) & (1, 13, 8235), (2, 2, 0), (2, 3, 0), (3, 1, 0) \\
		5490   & (1, 12, 13725) & (2, 2, 2745), (2, 3, 0), (3, 1, 0) \\
		8235   & (1, 12, 16470) & (1, 13, 13725), (2, 2, 2745), (2, 3, 0), (3, 1, 0) \\
		% <data removed>
		164700 & (1, 13, 13725) & (2, 2, 2745), (2, 3, 0), (3, 1, 0) \\
		0      & (1, 11, 13725) & (1, 12, 10980), (1, 13, 8235), (2, 2, 0), (3, 1, 0) \\
		2745   & (1, 12, 10980) & (1, 13, 8235), (2, 2, 0), (2, 3, 0), (3, 1, 0) \\
		5490   & (1, 12, 13725) & (2, 2, 2745), (2, 3, 0), (3, 1, 0) \\
		8235   & (1, 12, 16470) & (1, 13, 13725), (2, 2, 2745), (2, 3, 0), (3, 1, 0) \\
		% <data removed>
		164700 & (1, 13, 13725) & (2, 2, 2745), (2, 3, 0), (3, 1, 0) \\
		0      & (1, 11, 13725) & (1, 12, 10980), (1, 13, 8235), (2, 2, 0), (3, 1, 0) \\
		2745   & (1, 12, 10980) & (1, 13, 8235), (2, 2, 0), (2, 3, 0), (3, 1, 0) \\
		5490   & (1, 12, 13725) & (2, 2, 2745), (2, 3, 0), (3, 1, 0) \\
		8235   & (1, 12, 16470) & (1, 13, 13725), (2, 2, 2745), (2, 3, 0), (3, 1, 0) \\
		% <data removed>
		164700 & (1, 13, 13725) & (2, 2, 2745), (2, 3, 0), (3, 1, 0) \\
		0      & (1, 11, 13725) & (1, 12, 10980), (1, 13, 8235), (2, 2, 0), (3, 1, 0) \\
		2745   & (1, 12, 10980) & (1, 13, 8235), (2, 2, 0), (2, 3, 0), (3, 1, 0) \\
		5490   & (1, 12, 13725) & (2, 2, 2745), (2, 3, 0), (3, 1, 0) \\
		8235   & (1, 12, 16470) & (1, 13, 13725), (2, 2, 2745), (2, 3, 0), (3, 1, 0) \\
		% <data removed>
		164700 & (1, 13, 13725) & (2, 2, 2745), (2, 3, 0), (3, 1, 0) \\
		0      & (1, 11, 13725) & (1, 12, 10980), (1, 13, 8235), (2, 2, 0), (3, 1, 0) \\
		2745   & (1, 12, 10980) & (1, 13, 8235), (2, 2, 0), (2, 3, 0), (3, 1, 0) \\
		5490   & (1, 12, 13725) & (2, 2, 2745), (2, 3, 0), (3, 1, 0) \\
		8235   & (1, 12, 16470) & (1, 13, 13725), (2, 2, 2745), (2, 3, 0), (3, 1, 0) \\ \noalign{\penalty-5000}
		% <data removed>
		164700 & (1, 13, 13725) & (2, 2, 2745), (2, 3, 0), (3, 1, 0) \\
		0      & (1, 11, 13725) & (1, 12, 10980), (1, 13, 8235), (2, 2, 0), (3, 1, 0) \\
		2745   & (1, 12, 10980) & (1, 13, 8235), (2, 2, 0), (2, 3, 0), (3, 1, 0) \\
		5490   & (1, 12, 13725) & (2, 2, 2745), (2, 3, 0), (3, 1, 0) \\
		8235   & (1, 12, 16470) & (1, 13, 13725), (2, 2, 2745), (2, 3, 0), (3, 1, 0) \\
		% <data removed>
		164700 & (1, 13, 13725) & (2, 2, 2745), (2, 3, 0), (3, 1, 0) \\
		0      & (1, 11, 13725) & (1, 12, 10980), (1, 13, 8235), (2, 2, 0), (3, 1, 0) \\
		2745   & (1, 12, 10980) & (1, 13, 8235), (2, 2, 0), (2, 3, 0), (3, 1, 0) \\
		5490   & (1, 12, 13725) & (2, 2, 2745), (2, 3, 0), (3, 1, 0) \\
		8235   & (1, 12, 16470) & (1, 13, 13725), (2, 2, 2745), (2, 3, 0), (3, 1, 0) \\
		% <data removed>
		164700 & (1, 13, 13725) & (2, 2, 2745), (2, 3, 0), (3, 1, 0) \\
		0      & (1, 11, 13725) & (1, 12, 10980), (1, 13, 8235), (2, 2, 0), (3, 1, 0) \\
		2745   & (1, 12, 10980) & (1, 13, 8235), (2, 2, 0), (2, 3, 0), (3, 1, 0) \\
		5490   & (1, 12, 13725) & (2, 2, 2745), (2, 3, 0), (3, 1, 0) \\
		8235   & (1, 12, 16470) & (1, 13, 13725), (2, 2, 2745), (2, 3, 0), (3, 1, 0) \\
		% <data removed>
		164700 & (1, 13, 13725) & (2, 2, 2745), (2, 3, 0), (3, 1, 0) \\
		0      & (1, 11, 13725) & (1, 12, 10980), (1, 13, 8235), (2, 2, 0), (3, 1, 0) \\
		2745   & (1, 12, 10980) & (1, 13, 8235), (2, 2, 0), (2, 3, 0), (3, 1, 0) \\
		5490   & (1, 12, 13725) & (2, 2, 2745), (2, 3, 0), (3, 1, 0) \\
		8235   & (1, 12, 16470) & (1, 13, 13725), (2, 2, 2745), (2, 3, 0), (3, 1, 0) \\
		% <data removed>
		164700 & (1, 13, 13725) & (2, 2, 2745), (2, 3, 0), (3, 1, 0) \\
		0      & (1, 11, 13725) & (1, 12, 10980), (1, 13, 8235), (2, 2, 0), (3, 1, 0) \\
		2745   & (1, 12, 10980) & (1, 13, 8235), (2, 2, 0), (2, 3, 0), (3, 1, 0) \\
		5490   & (1, 12, 13725) & (2, 2, 2745), (2, 3, 0), (3, 1, 0) \\
		8235   & (1, 12, 16470) & (1, 13, 13725), (2, 2, 2745), (2, 3, 0), (3, 1, 0) \\
		% <data removed>
		164700 & (1, 13, 13725) & (2, 2, 2745), (2, 3, 0), (3, 1, 0) \\
		0      & (1, 11, 13725) & (1, 12, 10980), (1, 13, 8235), (2, 2, 0), (3, 1, 0) \\
		2745   & (1, 12, 10980) & (1, 13, 8235), (2, 2, 0), (2, 3, 0), (3, 1, 0) \\
		5490   & (1, 12, 13725) & (2, 2, 2745), (2, 3, 0), (3, 1, 0) \\
		8235   & (1, 12, 16470) & (1, 13, 13725), (2, 2, 2745), (2, 3, 0), (3, 1, 0) \\
	\end{longtable}

\section{Fault Diagnosis Techniques}

\subsection{Model based condition monitoring}
\subsubsection{State estimation}
\subsubsection{Residual generation}
\subsubsection{Identification}

\subsection{Model free condition monitoring}
\subsubsection{Signal analysis}
\subsubsection{Classical machine learning methods}
\subsubsubsection{Support Vector Machines}
\subsubsubsection{Naive Bayes}
\subsubsubsection{k-Nearest Neighbour}
\subsubsubsection{Random Forest}
\subsubsubsection{Multi Layer Perceptron}
\subsubsection{Deep learning methods}	
\subsubsubsection{1D Convolutional Neural Networks}
\subsubsubsection{Long-Short Term Memory Networks}
	
\end{center}