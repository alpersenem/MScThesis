%%%%%%%%%%%%%%%%%%%%%%%%%%%%%%%%%%%%%%%%%%%%%%%%%%%%%%%%%%%%%%%%%
\chapter{CONDITION MONITORING OF INDUCTION MOTORS: BACKGROUND }\label{Ch2}
%%%%%%%%%%%%%%%%%%%%%%%%%%%%%%%%%%%%%%%%%%%%%%%%%%%%%%%%%%%%%%%%%
\vspace*{-12pt} % If no text above section, use this vspace* to lift the whole part to the proper starting point - SBÖ
\section{Introduction of Induction Motors}
\subsection{Principle of operation}

Electric motors are divided into two classes depending on their power supply type: direct current (DC) or alternating current (AC). The latter can be broken into two classes as synchronous or induction according to their operating speed. Induction motors, which operates slightly lower than synchronous speed, are also sub-divided as wounded and squirrel-cage motors. In this study, squirrel-cage induction motors have been investigated by means of induction motors, since the squirrel-cage type is predominantly used in industrial applications. 

Induction motors run at a speed slightly lower than synchronous speed at the point where motor torque and load torque are equal [manual for industrial]. The difference between the actual speed and synchronous speed is known as slip [sourcebook].

synchronous speed equation comes here.

slip equation comes here.

In Principle, induction motors transfer electrical energy into mechanical energy by interlinking two electrical components: stator as stationary part and rotor as rotational part. Electrical energy transmitted from stator to rotor via electromagnetic induction, then a mechanical component bearing guides rotor to provide mechanical power [electric motor, induction motor fault diag].

motor diagram comes here.

\subsection{VFD-fed induction motors}

two figures will be added from iet 108.

A variable frequency drive, also called as adjustable-frequency drive (AFD), variable-speed drive (VSD) or inverter, fed motor system controls the rotation speed of the induction motor by controlling the supply frequency and voltage of the motor. The main difference between line-start and VFD-fed induction motors is that while in line-start mode supply voltage is the only controllable parameter, on the other hand, VFD-fed has the ability to control torque and speed easily [IET energy].

From a historical point of view, DC motors have been utilised in speed control applications. However, as a result of advances in power semiconductor technology used in inverters, the performance of AC motors in terms of precision, response, and speed range began to exceed that of DC motors [improving, historical]. As a driving force behind the induction motor control dominance today, VFDs generally have the following control strategies regarding speed and torque regulation:

\begin{itemize}
	\item Voltage per Frequency Control (V/f)
	\item Field Oriented Control (FOC)
	\item Direct Torque Control (DTC)
\end{itemize}

The common idea behind these methods is based on controlling the torque and flux references applied to the motor separately, as in DC motor control [iet energy]. In the scope of this thesis, only the V/f control strategy emphasized due to the widespread adoption of the control method in pump, compressor and fan applications. 

V/f control can be employed in both open-loop and closed-loop modes. Open-loop V/f control, which is by far the most popular control due to its simplicity, as the name implies, creates a constant air-gap flux by keeping the ratio between the voltage and frequency applied to the induction motor constant, and as a result, it provides the opportunity to work at operating frequencies from zero to nominal frequency [bose]. 

VFDs come with benefits such that energy savings, reliability and product quality, yet in concern of fault diagnosis they introduce number of factors,which will be discussed later on, that increase the complexity. 

\subsection{Need for condition monitoring}
\subsection{Maintenance strategies}

\vspace{6pt}
\begin{figure}[h]
	\centering
	\includegraphics[scale=.3]{./fig/sekil1}
	% sekil1.eps: 0x0 pixel, 300dpi, 0.00x0.00 cm, bb=14 14 592 479
	\vspace{6pt}
	\caption{All tables and figures must be horizontally centered on the page.}
	\label{Figure2.1}
\end{figure}

%\begin{figure}
%	\begin{minipage}[b]{.5\linewidth}
%		\centering
%		\includegraphics[scale=.2]{./fig/sekil1}
%		\subcaption{A subfigure}\label{Figure2.2a}
%	\end{minipage}
%	\begin{minipage}[b]{.5\linewidth}
%		\centering
%		\includegraphics[scale=.2]{./fig/sekil1}
%		\subcaption{Another subfigure}\label{Figure2.2b}
%	\end{minipage}
%	\caption{A figure}\label{Figure2.2} % If no need a caption for main figure comment it out 
%\end{figure}
%%Figure letter: \subref{Figure2.2a}

%\begin{figure}
%	\begin{subfigure}[b]{.5\linewidth}
%		\centering
%		\includegraphics[scale=.2]{./fig/sekil1}
%		\caption{A subfigure}\label{Figure2.3a}
%	\end{subfigure}
%	\begin{subfigure}[b]{.5\linewidth}
%		\centering
%		\includegraphics[scale=.2]{./fig/sekil1}
%		\caption{Another subfigure}\label{Figure2.3b}
%	\end{subfigure}
%	\caption{A figure}\label{Figure2.3}
%\end{figure}

% Subfigure example with proper LOF usage - SBÖ
\begin{figure}[h]
	\centering
	\begin{subfigure}{.8\textwidth}
		\centering
		\includegraphics[scale=.3]{./fig/sekil1}
		\firstsubcaption{First subcaption of the subfigure.}
		\label{Figure2.2a}
	\end{subfigure}
	\begin{subfigure}{.8\textwidth}
		\centering
		\includegraphics[scale=.3]{./fig/sekil1}
		\nextsubcaption{Second subcaption of the subfigure.}
		\label{Figure2.2b}
	\end{subfigure}
    \caption{An example of subfigure main caption.}\label{Figure2.2}
\end{figure}

%\begin{figure}
%	\centering     % not \center
%	\subcaption[]{Another subfigure}{\label{fig:a}\includegraphics[scale=.2]{./fig/sekil1}}
%	\subcaption[]{Another subfigure}{\label{fig:b}\includegraphics[scale=.2]{./fig/sekil1}}
%	%\caption{(a) this is fig 1 (b) this is fig 2.}
%	\label{L50}
%\end{figure}

In Figure \ref{Figure2.2}, sed diam nonumy eirmod tempor invidunt ut labore et dolore magna aliquyam erat, sed diam voluptua. At vero eos et accusam et justo duo dolores et ea rebum. Lorem ipsum dolor sit amet, consetetur sadipscing elitr, sed diam nonumy eirmod tempor invidunt ut labore et dolore magna aliquyam erat, sed diam voluptua. At vero eos et accusam et justo duo dolores et ea rebum. At vero eos et accusam et justo duo dolores et ea rebum. At vero eos et accusam et justo duo dolores et ea rebum. At vero eos et accusam et justo duo dolores et ea rebum. At vero eos et accusam et justo duo dolores et ea rebum. At vero eos et accusam et justo duo dolores et ea rebum. At vero eos et accusam et justo duo dolores et ea rebum in Figure \ref{Figure2.2a}.

\begin{figure}
	\centering
	\includegraphics[width=10cm,keepaspectratio=true]{./fig/sekil2}
	% sekil2.eps: 0x0 pixel, 300dpi, 0.00x0.00 cm, bb=14 14 818 556
	\vspace{3pt}
	\caption{Example figure.}
	\label{Figure2.3}
\end{figure}
\vspace{-6pt}
\section{Induction Motor Fault Types}

\subsection{Bearing related faults}
\subsection{Stator related faults}
\subsection{Rotor related faults}

Lorem ipsum dolor sit amet, consetetur sadipscing elitr, sed diam nonumy eirmod tempor invidunt ut labore et dolore magna aliquyam erat, sed diam voluptua. At vero eos et accusam et justo duo dolores et ea rebum (Figure \ref{Figure2.3}). Lorem ipsum dolor sit amet, consetetur sadipscing elitr, sed diam nonumy eirmod tempor invidunt ut labore et dolore magna aliquyam erat, sed diam voluptua. At vero eos et accusam et justo duo dolores et ea rebum. 

Lorem ipsum dolor sit amet, consetetur sadipscing elitr, sed diam nonumy eirmod tempor invidunt ut labore et dolore magna aliquyam erat, sed diam voluptua. At vero eos et accusam et justo duo dolores et ea rebum. Lorem ipsum dolor sit amet, consetetur sadipscing elitr, sed diam nonumy eirmod tempor invidunt ut labore et dolore magna aliquyam erat, sed diam voluptua. At vero eos et accusam et justo duo dolores et ea rebum. Lorem ipsum dolor sit amet, consetetur sadipscing elitr, sed diam nonumy eirmod tempor invidunt ut labore et dolore magna aliquyam erat, sed diam voluptua. At vero eos et accusam et justo duo dolores et ea rebum \cite{Deci_Ryan_1990}. 

Lorem ipsum dolor sit amet, consetetur sadipscing elitr, sed diam nonumy eirmod tempor invidunt ut labore et dolore magna aliquyam erat, sed diam voluptua. At vero eos et accusam et justo duo dolores et ea rebum. Lorem ipsum dolor sit amet, consetetur sadipscing elitr, sed diam nonumy eirmod tempor invidunt ut labore et dolore magna aliquyam erat, sed diam voluptua. At vero eos et accusam et justo duo dolores et ea rebum. Lorem ipsum dolor sit amet, consetetur sadipscing elitr, sed diam nonumy eirmod tempor invidunt ut labore et dolore magna aliquyam erat, sed diam voluptua. At vero eos et accusam et justo duo dolores et ea rebum. Lorem ipsum dolor sit amet, consetetur sadipscing elitr, sed diam nonumy eirmod tempor invidunt ut labore et dolore magna aliquyam erat, sed diam voluptua. At vero eos et accusam et justo duo dolores et ea rebum \cite{lepichon}.

% Change margins on the fly to reset the page margins to one inch - SBÖ
\newenvironment{changemargin}[4]{
	\begin{list}{}{
			\setlength{\voffset}{#1}
			\setlength{\oddsidemargin}{#2}
			\setlength{\evensidemargin}{#3}
			\setlength{\textheight}{#3}
		}
		\item[] ~ \par
		% Get rid of the extra space inserted by the previous line
		%\vspace*{-2em}
	}{
	\end{list}
}

% All the figures and also odd page figures normally face inside the thesis, however the rule requires figures always face to the right. - SBÖ
% Figures on landscape pages has to be centered and facing to the right (ITU) - SBÖ
\begin{landscape}
	\thispagestyle{empty} %Remove the bottom page numbering
%	\begin{changemargin}{-0.4mm}{0mm}{0mm} %Set all the margins to zero - SBÖ
	%\thispagestyle{lscape}
	\vspace*{5mm}
	\begin{figure*}[ht]
		\centering
		%\begin{tabular}{@{}cc@{}}
		\includegraphics[scale=.41,keepaspectratio=true]{./fig/sekil3} %&
		%\includegraphics[width=50mm]{./fig/sekil3}
		%\end{tabular}                                       
		\caption{Landscape-oriented, full-page figure.}
		\label{Figure2.4}
	\end{figure*}
	
% Set the page number on the right side for odd numbered pages
      \begin{tikzpicture}[remember picture, overlay]
		\node[xshift=-25mm+148.5mm, yshift=17mm-210mm+15mm] (number) at (current page text area.east) {\thepage};
	  \end{tikzpicture}
	  
%\end{changemargin}
\end{landscape}

% All the figures and also even page figures normally face inside the thesis, however the rule requires figures always face to the right. - SBÖ
% Figures on landscape pages has to be centered and facing to the right (ITU) - SBÖ
\begin{landscape}
	\thispagestyle{empty} % Remove the bottom page numbering
%	\begin{changemargin}{-0.4mm}{0mm}{0mm} %Set all the margins to zero - SBÖ
		%\thispagestyle{lscape}

		\vspace*{20mm}
		\begin{figure*}[ht]
			\centering
			%\begin{tabular}{@{}cc@{}}
				\includegraphics[scale=.41,keepaspectratio=true]{./fig/sekil3} %&
				%\includegraphics[width=50mm]{./fig/sekil3}
			%\end{tabular}                                       
			\caption{Landscape-oriented, full-page figure.}
			\label{Figure2.5}
		\end{figure*}
	   
% Set the page number on the left side for even numbered pages
		%\begin{tikzpicture}[remember picture, overlay]
		% \node[xshift=-25mm+148.5mm, yshift=-1mm-15mm, rotate=180] (number) at (current page text area.east) {\thepage};
		%\end{tikzpicture}
		
% Set the page number on the right side for even numbered pages as well
		\begin{tikzpicture}[remember picture, overlay]
		 \node[xshift=-25mm+148.5mm, yshift=17mm-210mm] (number) at (current page text area.east) {\thepage};
		\end{tikzpicture}
		
%	\end{changemargin}
\end{landscape}

%\newpage
\section{Condition Monitoring Techniques}

\subsection{Temperature monitoring}
\subsection{Vibration monitoring}
\subsection{Motor current monitoring}


Lorem ipsum dolor sit amet, consetetur sadipscing elitr, sed diam nonumy eirmod tempor invidunt ut labore et dolore magna aliquyam erat, sed diam voluptua. At vero eos et accusam et justo duo dolores et ea rebum. Stet clita kasd gub rgren, no sea takimata sanctus est Lorem ipsum dolor sit amet, consetetur sadipscing elitr, sed diam nonumy eirmod tempor invidunt ut lab ore sit et dolore magna.

\begin{table*}[h]
	{\setlength{\tabcolsep}{14pt}
		\caption{Table with single row and centered columns.}
		\begin{center}
			\vspace{-6mm}
			\begin{tabular}{cccc}
				\hline \\[-2.45ex] \hline \\[-2.1ex]
				Column A & Column B & Column C & Column D \\
				\hline \\[-1.8ex]
				Row A & Row A & Row A & Row A \\
				Row B & Row B & Row B & Row B \\
				Row C & Row C & Row C & Row C \\
				\hline
			\end{tabular}
			\vspace{-6mm}
		\end{center}
		\label{Table2.1}}
\end{table*}

As seen in Table \ref{Table2.1}, lorem ipsum dolor sit amet, consetetur sadipscing elitr, sed diam nonumy eirmod tempor invidunt ut labore et dolore magna aliquyam erat, sed diam voluptua. At vero eos et accusam et justo duo dolores et ea rebum. Stet clita kasd gub rgren, no sea takimata sanctus est Lorem ipsum dolor sit amet, consetetur sadipscing elitr, sed diam nonumy eirmod tempor invidunt ut lab ore sit et dolore magna.

\begin{table*}[h]
	{\setlength{\tabcolsep}{14pt}
		\caption{Table captions must be ended with a full stop.}
		\begin{center}
			\vspace{-6mm}
			\begin{tabular}{cccc}
				\hline \\[-2.45ex] \hline \\[-2.1ex]
				Column A & Column B & Column C & Column D \\
				\hline \\[-1.8ex]
				Row A & Row A & Row A & Row A \\
				Row B & Row B & Row B & Row B \\
				Row C & Row C & Row C & Row C \\
				\hline
			\end{tabular}
			\vspace{-6mm}
		\end{center}
		\label{Table2.2}}
\end{table*}

Lorem ipsum dolor sit amet, consetetur sadipscing elitr, sed diam nonumy eirmod tempor invidunt ut labore et dolore magna aliquyam erat, sed diam voluptua. At vero eos et accusam et justo duo dolores et ea rebum, as seen in Table \ref{Table2.2}. 

Lorem ipsum dolor sit amet, consetetur sadipscing elitr, sed diam nonumy eirmod tempor invidunt ut labore et dolore magna aliquyam erat, sed diam voluptua. At vero eos et accusam et justo duo dolores et ea rebum. Stet clita kasd gub rgren, no sea takimata sanctus est Lorem ipsum dolor sit amet, consetetur sadipscing elitr, sed diam nonumy eirmod tempor invidunt ut lab ore sit et dolore magna. Lorem ipsum dolor sit amet, consetetur sadipscing elitr, sed diam nonumy eirmod tempor invidunt ut labore et dolore magna aliquyam erat, sed diam voluptua. At vero eos et accusam et justo duo dolores et ea rebum \cite{Roberts_Jackson_1991}. 

\section{Signal Processing Techniques}
\subsection{Time domain based signal analysis}
\subsubsection{Higher order statistics}
\subsection{Time-frequency based signal analysis}
\subsubsection{Wavelet Transform}
\subsection{Frequency based signal analysis}
\subsubsection{Shannon-Nyquist sampling theory}
\subsubsection{Fast Fourier transform}
\subsubsection{Power spectral density estimation}

Lorem ipsum dolor sit amet, consetetur sadipscing elitr, sed diam nonumy eirmod tempor invidunt ut labore et dolore magna aliquyam erat, sed diam voluptua. At vero eos et accusam et justo duo dolores et ea rebum. Stet clita kasd gub rgren, no sea takimata sanctus est Lorem ipsum dolor sit amet, consetetur sadipscing elitr, sed diam nonumy eirmod tempor invidunt ut lab ore sit et dolore magna.

% ---------------------------------------------------------------- %
% Page numbers must be on the bottom-middle of short side (when    %
% portrait-oriented), or bottom-middle of long side (when	       %
% landscape-oriented)						                       %
% ---------------------------------------------------------------- %
% Odd page landscape table and page numbering - SBÖ		
\begin{landscape}
	\thispagestyle{empty}
%	\vspace*{-6mm}
%	\begin{changemargin}{0.4mm}{0mm}{0mm} %Set all the margins to zero - SBÖ
	\begin{table*}[htb!]
		{\setlength{\tabcolsep}{14pt}
			%\hspace*{5mm}
			%\vspace*{-6mm}
			\caption{Prof. Dr. Galip TEPEHAN \,\, Captioning in landscape-oriented pages:
				the most important aspect is to align the lines horizontally.}
			\begin{center}
				\vspace{-6mm}
				\begin{tabular}{lccrrrrr}
					\hline\hline
					\multirow{2}{*}{Parametre} & \multirow{2}{*}{Column 2} & \multirow{2}{*}{Column 3} & \multicolumn{3}{c|}{Column 4} & \multicolumn{2}{c}{Column 5}\\ \cline{4-8}
					& & & Subcolumn & Subcolumn & Subcolumn & Subcolumn & Subcolumn\\
					\hline
					Row 1 & -7.680442 & 7.6986348 & 0.00 & 0.00 & 0.00 & 12 & 12 \\
					Row 2 & 140 & - & 0.50 & 0.00 & 0.00 & 0 & 0 \\
					Row 3 & 37.174357 & 37.16192697 & 0.00 & 0.00 & 0.00 & 0 & 24 \\
					Row 4 & 140 & - & 0.50 & 0.00 & 0.00 & 0 & 0 \\
					Row 5 & 37.174357 & 37.16192697 & 0.00 & 0.00 & 0.00 & 0 & 24 \\
					Row 6 & 140 & - & 0.50 & 0.00 & 0.00 & 0 & 0 \\
					Row 7 & 37.174357 & 37.16192697 & 0.00 & 0.00 & 0.00 & 0 & 24 \\
					Row 8 & 140 & - & 0.50 & 0.00 & 0.00 & 0 & 0 \\
					Row 9 & 37.174357 & 37.16192697 & 0.00 & 0.00 & 0.00 & 0 & 24 \\
					Row 10 & 140 & - & 0.50 & 0.00 & 0.00 & 0 & 0 \\
					Row 11 & 37.174357 & 37.16192697 & 0.00 & 0.00 & 0.00 & 0 & 24 \\
					Row 12 & 140 & - & 0.50 & 0.00 & 0.00 & 0 & 0 \\
					Row 13 & 37.174357 & 37.16192697 & 0.00 & 0.00 & 0.00 & 0 & 24 \\
					Row 14 & 140 & - & 0.50 & 0.00 & 0.00 & 0 & 0 \\
					Row 15 & 37.174357 & 37.16192697 & 0.00 & 0.00 & 0.00 & 0 & 24 \\
					\hline
				\end{tabular}
			\end{center}
			\begin{center}
				\label{Table2.3}
			\end{center}
		}
	\end{table*}
% Set the page number on the right side for odd numbered pages
		\begin{tikzpicture}[remember picture,overlay]
		\node[xshift=-10mm+148.5mm, yshift=2mm-210mm+30mm] (number) at (current page text area.east) {\thepage};
		\end{tikzpicture}
%   \end{changemargin}
\end{landscape}

% ---------------------------------------------------------------- %
% Page numbers must be on the bottom-middle of short side (when    %
% portrait-oriented), or bottom-middle of long side (when	       %
% landscape-oriented)						                       %
% ---------------------------------------------------------------- %
% Even page landscape table and page numbering - SBÖ		
\begin{landscape}
	\thispagestyle{empty}
	%\vspace*{-6mm}
%	\begin{changemargin}{0.4mm}{0mm}{0mm} %Set all the margins to zero - SBÖ
		\begin{table*}[htb!]
			{\setlength{\tabcolsep}{14pt}
				%\hspace*{5mm}
				%\vspace*{-6mm}
				\caption{Prof. Dr. Galip TEPEHAN \,\, Captioning in landscape-oriented pages:
					the most important aspect is to align the lines horizontally.}
				\begin{center}
					\vspace{-6mm}
					\begin{tabular}{lccrrrrr}
						\hline\hline
						\multirow{2}{*}{Parametre} & \multirow{2}{*}{Column 2} & \multirow{2}{*}{Column 3} & \multicolumn{3}{c|}{Column 4} & \multicolumn{2}{c}{Column 5}\\ \cline{4-8}
						& & & Subcolumn & Subcolumn & Subcolumn & Subcolumn & Subcolumn\\
						\hline
						Row 1 & -7.680442 & 7.6986348 & 0.00 & 0.00 & 0.00 & 12 & 12 \\
						Row 2 & 140 & - & 0.50 & 0.00 & 0.00 & 0 & 0 \\
						Row 3 & 37.174357 & 37.16192697 & 0.00 & 0.00 & 0.00 & 0 & 24 \\
						Row 4 & 140 & - & 0.50 & 0.00 & 0.00 & 0 & 0 \\
						Row 5 & 37.174357 & 37.16192697 & 0.00 & 0.00 & 0.00 & 0 & 24 \\
						Row 6 & 140 & - & 0.50 & 0.00 & 0.00 & 0 & 0 \\
						Row 7 & 37.174357 & 37.16192697 & 0.00 & 0.00 & 0.00 & 0 & 24 \\
						Row 8 & 140 & - & 0.50 & 0.00 & 0.00 & 0 & 0 \\
					\end{tabular}
				\end{center}
				\begin{center}
					\label{Table2.4}
				\end{center}
			}
		\end{table*}
% Set the page number on the right side for even numbered pages
		\begin{tikzpicture}[remember picture,overlay]
		\node[xshift=-25mm+148.5mm, yshift=2mm-210mm+15mm] (number) at (current page text area.east) {\thepage};
		\end{tikzpicture}
%	\end{changemargin}
\end{landscape}

\begin{table}[!htbp] \centering
	\caption{ Neighborhoods Visited }
	\vspace{-3mm}
	\label{}
	\begin{tabular}{@{\extracolsep{5pt}} llrrr} 
	\\[-1.8ex]\hline 
		\hline \\[-1.8ex] 
		\multicolumn{1}{c}{Variable} & \multicolumn{1}{c}{Values} & \multicolumn{1}{c}{Count} & \multicolumn{1}{c}{\%} & \multicolumn{1}{c}{Cum. \%} \\
		\hline \\[-1.8ex] 
		\multirow{ 4 }{*}{ visit }  &  FALSE  &  2  &  33.33  &  33.33  \\
		\hhline{}  &  TRUE  &  3  &  50.00  &  83.33  \\
		\hhline{}  &  NA  &  1  &  16.67  &  100.00  \\
	    \hhline{}  &  Total  &  6  &  100.00  &    \\
		\hline \\[-1.8ex] 
	\end{tabular}
\end{table}

% Multi-page longtable example spreading couple of pages - SBÖ
\begin{center}
	\begin{longtable}{ccc}
		%Here is the caption, the stuff in [] is the table of contents entry,
		%the stuff in {} is the title that will appear on the first page of the
		%table.
		\caption[Feasible triples for a highly variable Grid]{Feasible triples
			for highly variable Grid, MLMMH.} \label{Table2.6} \vspace{-1.75mm}\\
		%This is the header for the first page of the table...
		\hline\\[-2.45ex] \hline \\[-1.8ex] % Distancing of the hlines adjausted from the text 
		\multicolumn{1}{c}{{Time (s)}} &
		\multicolumn{1}{c}{{Triple chosen}} &
		\multicolumn{1}{c}{{Other feasible triples}} \\[0.5ex] \hline
		\\[-1.8ex]
		\endfirsthead
		
		%This is the header for the remaining page(s) of the table...
		\multicolumn{3}{c}{{\tablename} \textbf{\thetable{}} \textbf{(continued) :} Feasible triples
			for highly variable Grid, MLMMH.} \\[0.5ex]
		\hline\\[-2.45ex] \hline \\[-1.8ex]
		\multicolumn{1}{c}{{Time (s)}} &
		\multicolumn{1}{c}{{Triple chosen}} &
		\multicolumn{1}{c}{{Other feasible triples}} \\[0.5ex] \hline
		\\[-1.8ex]
		\endhead
		
		%This is the footer for all pages except the last page of the table...
		%\multicolumn{3}{l}{{Continued on Next Page\ldots}} \\
		\\[-1.8ex] \hline
		\endfoot
		
		%This is the footer for the last page of the table...
		\\[-1.8ex] \hline
		\endlastfoot
		
		%Now the data...
		0      & (1, 11, 13725) & (1, 12, 10980), (1, 13, 8235), (2, 2, 0), (3, 1, 0) \\
		2745   & (1, 12, 10980) & (1, 13, 8235), (2, 2, 0), (2, 3, 0), (3, 1, 0) \\
		5490   & (1, 12, 13725) & (2, 2, 2745), (2, 3, 0), (3, 1, 0) \\
		8235   & (1, 12, 16470) & (1, 13, 13725), (2, 2, 2745), (2, 3, 0), (3, 1, 0) \\
		% <data removed>
		164700 & (1, 13, 13725) & (2, 2, 2745), (2, 3, 0), (3, 1, 0) \\
		0      & (1, 11, 13725) & (1, 12, 10980), (1, 13, 8235), (2, 2, 0), (3, 1, 0) \\
		2745   & (1, 12, 10980) & (1, 13, 8235), (2, 2, 0), (2, 3, 0), (3, 1, 0) \\
		5490   & (1, 12, 13725) & (2, 2, 2745), (2, 3, 0), (3, 1, 0) \\
		8235   & (1, 12, 16470) & (1, 13, 13725), (2, 2, 2745), (2, 3, 0), (3, 1, 0) \\
		% <data removed>
		164700 & (1, 13, 13725) & (2, 2, 2745), (2, 3, 0), (3, 1, 0) \\
		0      & (1, 11, 13725) & (1, 12, 10980), (1, 13, 8235), (2, 2, 0), (3, 1, 0) \\
		2745   & (1, 12, 10980) & (1, 13, 8235), (2, 2, 0), (2, 3, 0), (3, 1, 0) \\
		5490   & (1, 12, 13725) & (2, 2, 2745), (2, 3, 0), (3, 1, 0) \\
		8235   & (1, 12, 16470) & (1, 13, 13725), (2, 2, 2745), (2, 3, 0), (3, 1, 0) \\
		% <data removed>
		164700 & (1, 13, 13725) & (2, 2, 2745), (2, 3, 0), (3, 1, 0) \\
		0      & (1, 11, 13725) & (1, 12, 10980), (1, 13, 8235), (2, 2, 0), (3, 1, 0) \\
		2745   & (1, 12, 10980) & (1, 13, 8235), (2, 2, 0), (2, 3, 0), (3, 1, 0) \\
		5490   & (1, 12, 13725) & (2, 2, 2745), (2, 3, 0), (3, 1, 0) \\
		8235   & (1, 12, 16470) & (1, 13, 13725), (2, 2, 2745), (2, 3, 0), (3, 1, 0) \\
		% <data removed>
		164700 & (1, 13, 13725) & (2, 2, 2745), (2, 3, 0), (3, 1, 0) \\
		0      & (1, 11, 13725) & (1, 12, 10980), (1, 13, 8235), (2, 2, 0), (3, 1, 0) \\
		2745   & (1, 12, 10980) & (1, 13, 8235), (2, 2, 0), (2, 3, 0), (3, 1, 0) \\
		5490   & (1, 12, 13725) & (2, 2, 2745), (2, 3, 0), (3, 1, 0) \\
		8235   & (1, 12, 16470) & (1, 13, 13725), (2, 2, 2745), (2, 3, 0), (3, 1, 0) \\
		% <data removed>
		164700 & (1, 13, 13725) & (2, 2, 2745), (2, 3, 0), (3, 1, 0) \\
		0      & (1, 11, 13725) & (1, 12, 10980), (1, 13, 8235), (2, 2, 0), (3, 1, 0) \\
		2745   & (1, 12, 10980) & (1, 13, 8235), (2, 2, 0), (2, 3, 0), (3, 1, 0) \\
		5490   & (1, 12, 13725) & (2, 2, 2745), (2, 3, 0), (3, 1, 0) \\
		8235   & (1, 12, 16470) & (1, 13, 13725), (2, 2, 2745), (2, 3, 0), (3, 1, 0) \\
		% <data removed>
		164700 & (1, 13, 13725) & (2, 2, 2745), (2, 3, 0), (3, 1, 0) \\
		0      & (1, 11, 13725) & (1, 12, 10980), (1, 13, 8235), (2, 2, 0), (3, 1, 0) \\
		2745   & (1, 12, 10980) & (1, 13, 8235), (2, 2, 0), (2, 3, 0), (3, 1, 0) \\
		5490   & (1, 12, 13725) & (2, 2, 2745), (2, 3, 0), (3, 1, 0) \\
		8235   & (1, 12, 16470) & (1, 13, 13725), (2, 2, 2745), (2, 3, 0), (3, 1, 0) \\ \noalign{\penalty-5000}
		% <data removed>
		164700 & (1, 13, 13725) & (2, 2, 2745), (2, 3, 0), (3, 1, 0) \\
		0      & (1, 11, 13725) & (1, 12, 10980), (1, 13, 8235), (2, 2, 0), (3, 1, 0) \\
		2745   & (1, 12, 10980) & (1, 13, 8235), (2, 2, 0), (2, 3, 0), (3, 1, 0) \\
		5490   & (1, 12, 13725) & (2, 2, 2745), (2, 3, 0), (3, 1, 0) \\
		8235   & (1, 12, 16470) & (1, 13, 13725), (2, 2, 2745), (2, 3, 0), (3, 1, 0) \\
		% <data removed>
		164700 & (1, 13, 13725) & (2, 2, 2745), (2, 3, 0), (3, 1, 0) \\
		0      & (1, 11, 13725) & (1, 12, 10980), (1, 13, 8235), (2, 2, 0), (3, 1, 0) \\
		2745   & (1, 12, 10980) & (1, 13, 8235), (2, 2, 0), (2, 3, 0), (3, 1, 0) \\
		5490   & (1, 12, 13725) & (2, 2, 2745), (2, 3, 0), (3, 1, 0) \\
		8235   & (1, 12, 16470) & (1, 13, 13725), (2, 2, 2745), (2, 3, 0), (3, 1, 0) \\
		% <data removed>
		164700 & (1, 13, 13725) & (2, 2, 2745), (2, 3, 0), (3, 1, 0) \\
		0      & (1, 11, 13725) & (1, 12, 10980), (1, 13, 8235), (2, 2, 0), (3, 1, 0) \\
		2745   & (1, 12, 10980) & (1, 13, 8235), (2, 2, 0), (2, 3, 0), (3, 1, 0) \\
		5490   & (1, 12, 13725) & (2, 2, 2745), (2, 3, 0), (3, 1, 0) \\
		8235   & (1, 12, 16470) & (1, 13, 13725), (2, 2, 2745), (2, 3, 0), (3, 1, 0) \\
		% <data removed>
		164700 & (1, 13, 13725) & (2, 2, 2745), (2, 3, 0), (3, 1, 0) \\
		0      & (1, 11, 13725) & (1, 12, 10980), (1, 13, 8235), (2, 2, 0), (3, 1, 0) \\
		2745   & (1, 12, 10980) & (1, 13, 8235), (2, 2, 0), (2, 3, 0), (3, 1, 0) \\
		5490   & (1, 12, 13725) & (2, 2, 2745), (2, 3, 0), (3, 1, 0) \\
		8235   & (1, 12, 16470) & (1, 13, 13725), (2, 2, 2745), (2, 3, 0), (3, 1, 0) \\
		% <data removed>
		164700 & (1, 13, 13725) & (2, 2, 2745), (2, 3, 0), (3, 1, 0) \\
		0      & (1, 11, 13725) & (1, 12, 10980), (1, 13, 8235), (2, 2, 0), (3, 1, 0) \\
		2745   & (1, 12, 10980) & (1, 13, 8235), (2, 2, 0), (2, 3, 0), (3, 1, 0) \\
		5490   & (1, 12, 13725) & (2, 2, 2745), (2, 3, 0), (3, 1, 0) \\
		8235   & (1, 12, 16470) & (1, 13, 13725), (2, 2, 2745), (2, 3, 0), (3, 1, 0) \\
		% <data removed>
		164700 & (1, 13, 13725) & (2, 2, 2745), (2, 3, 0), (3, 1, 0) \\
		0      & (1, 11, 13725) & (1, 12, 10980), (1, 13, 8235), (2, 2, 0), (3, 1, 0) \\
		2745   & (1, 12, 10980) & (1, 13, 8235), (2, 2, 0), (2, 3, 0), (3, 1, 0) \\
		5490   & (1, 12, 13725) & (2, 2, 2745), (2, 3, 0), (3, 1, 0) \\
		8235   & (1, 12, 16470) & (1, 13, 13725), (2, 2, 2745), (2, 3, 0), (3, 1, 0) \\
	\end{longtable}

\section{Fault Diagnosis Techniques}

\subsection{Model based condition monitoring}
\subsubsection{State estimation}
\subsubsection{Residual generation}
\subsubsection{Identification}

\subsection{Model free condition monitoring}
\subsubsection{Signal analysis}
\subsubsection{Classical machine learning methods}
\subsubsubsection{Support Vector Machines}
\subsubsubsection{Naive Bayes}
\subsubsubsection{k-Nearest Neighbour}
\subsubsubsection{Random Forest}
\subsubsubsection{Multi Layer Perceptron}
\subsubsection{Deep learning methods}	
\subsubsubsection{1D Convolutional Neural Networks}
\subsubsubsection{Long-Short Term Memory Networks}
	
\end{center}