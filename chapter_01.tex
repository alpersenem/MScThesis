%%%%%%%%%%%%%%%%%%%%%%%%%%%%%%%%%%%%%%%%%%%%%%%%%%%%%%%%%%%%%%%%%
\chapter{INTRODUCTION}\label{Ch1}
%%%%%%%%%%%%%%%%%%%%%%%%%%%%%%%%%%%%%%%%%%%%%%%%%%%%%%%%%%%%%%%%%
\section{Overview}
\label{overvıew}

Electric motors extensively employed in a system that converts electrical power into mechanical power in not only industrial applications but also residential, agricultural and transportation purposes. Taken together with systems they drive, electric motors use more than 40\% of all electricity consumption and almost twice as much as the next largest user lighting \cite{waide2011energy}. Considering only industrial usage, electric motors dominate and account for close to 70\% of the total electricity consumption \cite{waide2011energy,kulterer2014policy}.

There are many different motor types available in industrial facility operations, but asynchronous alternating current (AC) induction motors are the most preferred type because of their simple, reliable and rugged design. Relatively lost cost, low maintenance, high reliability and long lifespan are the most advantageous features of AC induction motors which drive core electro-mechanical systems such as material handling, material processing, pumping, ventilation and compressed air generation \cite{Fleiter2012EnergyEI}. Especially HVAC (Heating, ventilation and air conditioning) sector requires special attention as they have the largest share of industrial electrical consumption and reasonably high saving potentials \cite{Fleiter2012EnergyEI}.

In recent years raised awareness about global warming demands more efficient systems including electric motor-driven systems. Policymakers such as the European Parliament and the European Council implementing new requirements to increase efficiency by encouraging the usage of high-efficiency premium motors and variable frequency drives (VFD) \cite{kulterer2014policy,mikami2011historical}.

VFDs regulate the motor's output torque and speed to match the mechanical system loads and enables significant energy efficiency where variable mechanical power needed that have highly non-linear input power, output torque and speed such as pumps, fans and compressors. Previously Direct Current (DC) motors have been dominant for variable motor speed control, yet developments in semiconductor technology became the driving force behind the prevalence usage of VFDs with AC motors \cite{doe2008improving}. Motor speed control is advantageous in terms of lower system energy costs, increased system reliability and less maintenance.  

Considering 20-year in service, the power consumption of an electric motor depicts 90\% of the total cost of ownership and followed by downtime costs as 5\% and rebuild costs as 4\% \cite{waide2011energy}. The initial purchase price represents only 1\% of the total cost and it can be concluded that savings can be achieved by actions taken during operation of motor \cite{waide2011energy}.

Industry 4.0 is shaping industrial operations through automation and efficiency. Condition monitoring paves the way to Industry 4.0 through evaluating the state of the plant and/or equipment throughout its service life \cite{en201713306}. Maintenance can be defined as actions to retain or restore equipment in order to maintain its designed functions within the entire lifespan \cite{en201713306}. Traditional maintenance relies on periodically health checks to provide operability, but researches show that even if maintenance is done on time and correctly the vast majority of failures arises during operation state \cite{motor1985report}. Condition monitoring and diagnostics can help to schedule maintenance to prevent such situations whilst avoiding unintended downtime and financial losses. Also, condition monitoring has the opportunity to build a database to understand better via trend analysis of the equipment or plant that leads more reliable system in the long run. 

There are many condition monitoring methods available such as vibration, temperature, and current monitoring that can be used to assess insights into the health of equipment varying from bearings to electric motors and pumps. Current monitoring distinguishes itself from other methods since it is readily measured to control induction motor operation. VFDs are presenting a great potential not only to control the motor operation but also to be utilised as a connection to the Internet of Things structure to serve Industry 4.0.

\section{Objectives of Research}
\label{objectives}

This study aims to diagnose and identify mechanical and electrical faults of VFD-fed induction motors under various loads and speeds via monitoring only motor current. As an outcome of this research comparative results among time-domain versus frequency-domain analysis and classical machine learning algorithms versus deep learning algorithms are presented.

The achievement of this study was facilitated by the following specific objectives:
\begin{itemize}
\item Analyse motor faults under VFD controlled motor current
\item Investigate effects of various loads and speeds
\item Build different feature engineering methods 
\item Benchmark Classical ML and Deep Learning algorithms 
\end{itemize}

\section{Organization of Thesis}
\label{organization}

Thesis organised in five chapters to achieve aforementioned objectives;
\begin{itemize}
\item Chapter-2 provides an in-depth background to condition monitoring and fault diagnosis of AC induction motors including general information about induction motors, fault types, condition monitoring and signal processing techniques followed by fault diagnosis methods and performance metrics.

\item Chapter-3 presents the experimental testing system and used methodology.

\item Chapter-4 discusses the diagnostics of faults via two different approaches: classical machine learning and deep learning-based condition monitoring.

\item Chapter-5 remarks obtained results with different approaches and concludes with future recommendations.

\end{itemize}