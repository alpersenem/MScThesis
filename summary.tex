The impact of Industry 4.0 awakens the need for equipment and system-level condition monitoring, especially for asynchronous motors, which are the backbone of the industry, in parallel with the increasing search for efficiency in industrial operations and the importance of data. Although the number of asynchronous motors fed directly from the line still dominates, the number of motors driven by VFD is also increasing with the demand for efficiency and the regulations applied.

Within the scope of this thesis, the most common faults in asynchronous motors, bearing, stator winding short circuit, and broken rotor bar faults were created in the laboratory environment of WAT Motor Company facilities, condition monitoring and fault diagnosis studies were carried out. Studies on line-start asynchronous motors are widely applied in the literature. However, in this study, tests were carried out in two different loadings at different speeds between 30 Hz and 50 Hz at 75\% of the nominal and nominal load of the motor fed with the variable frequency drive.

With the obtained data, different methods and approaches are examined for motor condition monitoring and fault diagnosis. In the first method, the statistical features of the current signal are extracted for time-domain analysis. In the study conducted in the frequency domain, the statistical features of the frequency spectrum amplitudes obtained by estimating the Power Spectral Density with the Welch method were obtained in the same way. Finally, the amplitudes corresponding to the characteristic frequencies of the fault types in the frequency spectrum were calculated and the statistical properties of these amplitudes were extracted. With the features obtained by these three methods, machine learning classifiers were trained and motor fault diagnosis was performed. According to the results obtained, it is possible to achieve good results in fault diagnosis with statistical approaches. On the other hand, considering the industrial application conditions, electric motors are heavily exposed to disruptive effects. PSD estimation with the Welch method gives very robust results against disturbance effects due to its nature. The amplitude-based statistical approach obtained from the PSD frequency spectrum with error characteristic formulas overperformed the other two methods with high accuracy and precision in all metrics.

Increasing deep learning studies and applications in recent years find a place for themselves in the diagnosis of motor failures. As an alternative to feature extraction and signal processing problems that require time and expertise in machine learning methods, deep learning methods can offer an end-to-end solution. As an alternative to the data engineering and machine learning methods mentioned in the thesis, two deep learning methods, convolutional and recurrent neural networks structured for fault diagnosis. Deep learning methods offer high performance without any preprocessing, but as a disadvantage, they require high processing power and a large dataset to train the model. Although deep learning methods need large data sets in the training phase, they can work with less dimensional data than machine learning methods, as shown in the thesis.