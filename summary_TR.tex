Elektrik motorları, sadece endüstriyel uygulamalarda değil, aynı zamanda konut, tarım ve ulaşım amaçlı olarak da elektrik gücünü mekanik güce dönüştüren sistemlerde yaygın olarak kullanılmaktadır. Sürdükleri sistemlerle birlikte ele alındığında, elektrik motorları tüm elektrik tüketiminin \%40'ından fazlasını ve bir sonraki en büyük tüketici olan aydınlatmanın neredeyse iki katı kadarını kullanır. Sadece endüstriyel kullanım düşünüldüğünde, elektrik motorları toplam elektrik tüketiminin \%70'ine yakınını oluşturmaktadır.

Endüstriyel uygulamalarda birçok farklı motor tipi mevcuttur ancak asenkron alternatif akım (AC) asenkron motorlar basit, güvenilir ve sağlam tasarımları nedeniyle en çok tercih edilen tiptir. Malzeme taşıma, malzeme işleme, pompalama, havalandırma ve basınçlı hava üretimi gibi temel elektro-mekanik sistemleri çalıştıran AC asenkron motorların nispeten kayıp maliyeti, düşük bakım, yüksek güvenilirlik ve uzun kullanım ömrü en avantajlı özellikleridir. Özellikle HVAC (Isıtma, havalandırma ve iklimlendirme) sektörü, endüstriyel elektrik tüketiminde en büyük paya sahip oldukları ve oldukça yüksek tasarruf potansiyellerine sahip oldukları için özel önem gerektirmektedir.

Son yıllarda küresel ısınma konusunda artan farkındalık, elektrik motorlu sistemler de dahil olmak üzere daha verimli sistemler talep ediyor. Avrupa Parlamentosu ve Avrupa Konseyi gibi otoriteler, yüksek verimli premium motorların ve değişken frekanslı sürücülerin (VFD) kullanımını teşvik ederek verimliliği artırmak için yeni gereksinimler uygulamaktadır. VFD'ler, motorun çıkış torkunu ve hızını mekanik sistem yüklerine uyacak şekilde düzenler ve pompalar, fanlar ve kompresörler gibi yüksek düzeyde doğrusal olmayan giriş gücüne ve çıkış torku ve hızına sahip değişken mekanik gücün gerekli olduğu yerlerde önemli enerji verimliliği sağlar. 

20 yıllık kullanım süresi göz önüne alındığında, bir elektrik motorunun güç tüketimi toplam sahip olma maliyetinin \%90'ını oluşturuken, bunu \%5 ile arıza süresi maliyeti ve \%4 ile bakım maliyeti izlemektedir. İlk satın alma fiyatı ise toplam maliyetin sadece \%1'ini temsil ettiği düşünüldüğünde, motorun çalışması sırasında alınan önlemlerle tasarruf sağlanabileceği sonucuna varılabilir. Bu noktada ise Endüstri 4.0, otomasyon ve verimlilik yoluyla endüstriyel operasyonları şekillendiriyor. Durum izleme, çalıştığı süre boyunca tesisin ve/veya ekipmanın durumunu değerlendirerek Endüstri 4.0'ın temel yapıtaşlarından birini oluşturmaktadır. 

Bakım, tüm kullanım ömrü boyunca tasarlanan işlevlerini sürdürmek için ekipmanı korumak veya eski haline getirmek için yapılan eylemler olarak tanımlanabilir. Geleneksel bakım, çalışabilirliği sağlamak için periyodik sağlık kontrollerine dayanır, ancak araştırmalar, bakım zamanında ve doğru bir şekilde yapılsa bile arızaların büyük çoğunluğunun çalışma durumunda ortaya çıktığını göstermektedir. Durum izleme ve arıza teşhisi, bu tür durumları önlemek için bakım planlamasına yardımcı olurken, istenmeyen arıza sürelerini ve mali kayıpları da önler. Ayrıca durum izleme, uzun vadede daha güvenilir bir sistem sağlayan ekipman veya tesisin trend analizi yoluyla daha iyi anlamak için bir veritabanı oluşturma fırsatına da sahiptir.

Durum izleme, arıza tespiti için bir teşhis aracı ve bakım planlamasının temellerinden biri olarak motora sürekli veya periyodik olarak uygulanır. İzlenen parametrelerdeki ani veya beklenmedik değişiklikler motorun durumu hakkında önemli bilgiler sağlar. Rulmanlardan elektrik motorlarına ve pompalara kadar çeşitli ekipmanların sağlığına ilişkin bilgileri değerlendirmek için kullanılabilecek titreşim, sıcaklık ve akım izleme gibi birçok durum izleme yöntemi mevcuttur. Akım izleme, asenkron motor çalışmasını kontrol etmek için kolayca ölçülebildiği için kendisini diğer yöntemlerden ayırmaktadır.

Besleme akımı sinyalleri aracılığıyla durum izleme, yalnızca motorun kendisi için değil aynı zamanda motorun tahrik ettiği mekanik sistem hakkında da faydalı bilgiler sağlar. Bakım stratejisinin önemli bir yönü, motorun tahrik ettiği mekanik bir sistemin dahil edilmesidir. Özellikle motorun çalışmasını kontrol etmek için akımın zaten algılandığı VFD beslemeli sistemlerde, ek sensör ihtiyacı olmadan hem elektriksel hem de mekanik arızalar teşhis edilebilir. Akım sinyallerini kullanarak arıza teşhisi için birçok çalışma yapılmış olmasına rağmen, VFD beslemeli motorlarla ilgili çalışmalar sınırlıdır. VFD'lerde kullanılan PWM sinyallerinin, motor akım sinyalindeki bir arızanın özelliklerini maskeleyerek, teşhisi zorlaştırabileceğine dikkat edilmelidir. Bu çalışmada, VFD beslemeli üç fazlı asenkron motorun tek fazlı stator besleme akımı üzerinden farklı yük ve frekans senaryolarında elektriksel ve mekanik arıza tespiti üzerinde durulmuştur.

Bu çalışmada öncelikle asenkron motorların endüstriyel alandaki yeri ve önemi ile genel çalışma prensibinden bahsedilmiştir. Literatür taraması kapsamında işletme ömürleri boyunca maruz kaldıkları stresler nedeniyle oluşan arızalar açıklanmıştır. Arızaların olumsuz sonuçlarından en az hasarla kaçınmak için etkin bakım yöntemlerinden bahsedilmiş ve verimli bakımın omurgası olarak durum izleme ve hata tanıma yöntemleri anlatılmıştır. Çalışma kapsamında asenkron motor arızalarının en sık görülen türleri olan rulman, stator sargı kısa devre ve kırık rotor çubuk arızaları WAT Motor tesislerinde laboratuvar koşullarında oluşturulmuş ve etkileri incelenmiştir. Literatürde besleme hattından doğrudan beslenen ve nominal hızda çalışan asenkron motorlar ile ilgili çalışmalar yaygındır. Bu çalışmada, değişken frekanslı sürücü ile beslenen motorun nominal ve nominal yükünün \%75'inde 30 Hz ile 50 Hz arasında farklı hızlarda iki farklı yüklemede testler gerçekleştirilmiştir.

Elde edilen veriler ile motor durum izleme ve arıza tespiti için farklı yöntem ve yaklaşımlar incelenmiştir. Bunlardan ilkinde, zaman domeninde yapılan analizde akım sinyalinin istatistiksel özellikleri çıkarılmıştır. Frekans alanında yapılan çalışmada ise, Welch yöntemi ile Güç Spektral Yoğunluğu tahmin edilerek elde edilen genliklerin istatistiksel özellikleri aynı şekilde elde edilmiştir. Son olarak frekans spektrumundaki arıza tiplerinin karakteristik frekanslarına karşılık gelen genlikler hesaplanmış ve bu genliklerin istatistiksel özellikleri çıkarılmıştır. Bu üç yöntemle elde edilen özellikler ile klasik makine öğrenmesi sınıflandırıcıları eğitilmiş ve motor arıza teşhisi gerçekleştirilmiştir. Elde edilen bulgulara göre istatistiksel yaklaşımlarla arıza teşhisi konusunda iyi sonuçlar almak mümkündür. Endüstriyel uygulama koşulları göz önüne alındığında elektrik motorları, bozucu etkilere yoğun bir şekilde maruz kalmaktadır. Welch yöntemi ile PSD tahmini ise, yapısı gereği bozucu etkilere karşı oldukça sağlam sonuçlar vermektedir. Hata karakteristik formülleri ile PSD frekans spektrumundan elde edilen genlik tabanlı istatistiksel yaklaşım, tüm metriklerde yüksek doğruluk ve kesinlik ile diğer iki yöntemi geride bırakmıştır.

Son yıllarda artan derin öğrenme çalışmaları ve uygulamaları, motor arızalarının teşhisinde de kendisine yer bulmaktadır. Klasik makine öğrenmesi yöntemlerinde zaman ve uzmanlık bilgisi gerektiren özellik çıkarma ve sinyal işleme problemlerine alternatif olarak derin öğrenme yöntemleri uçtan uca bir çözüm sunabilmektedir. Tez kapsamında bahsedilen veri mühendisliği ve klasik makine öğrenmesi yöntemlerine alternatif olarak evrişimli ve tekrarlayan sinir ağları olmak üzere iki derin öğrenme yöntemi ile hata teşhisi yapılmıştır. Derin öğrenme yöntemleri, herhangi bir ön işleme gerekmeden yüksek performans gösterir, ancak bir dezavantaj olarak, modeli eğitmek için yüksek işlem gücüne ve büyük bir veri kümesine ihtiyaç duyarlar. Derin öğrenme yöntemleri, eğitimleri sırasında büyük veri kümelerine ihtiyaç duysa da tez kapsamında gösterildiği gibi klasik makine öğrenmesi yöntemlerine göre daha az boyutlu verilerle çalışabilirler. Bu açıdan VFD'ye entegre edilecek eğitilmiş bir derin öğrenme algoritması, ileri beslemeli yapısı sayesinde daha az kaynak ihtiyacı ile çalışabilir.

Sonuç olarak Endüstri 4.0 kapsamında bazı kavramlar daha da önemli hale geliyor. Bunlardan ilk ikisi verilere erişimi kolaylaşması ve veri işleme gücününün artması iken, üçüncüsü verimlilik olarak sayılabilir. Asenkron motorlarda bu üç kavram VFD'ler ile karşılanabilir. VFD ile beslenen motorlarda sistemin verimi artarken, motor kontrolü için gerekli olan akım sinyali herhangi bir ek masraf olmadan durum izleme çalışmaları için kullanılabilir. Toplanan verileri Nesnelerin İnterneti altyapıları ile veri merkezlerine aktararak veri erişimini kolaylaştırma potansiyeline de sahiptir.
 
